\documentclass[12pt]{article}
\usepackage[T1]{fontenc}
\usepackage[T1]{polski}
\usepackage[utf8]{inputenc}
\usepackage{color}
\usepackage{graphicx}
\usepackage{blindtext}
\usepackage{scrextend}
\newcommand{\BibTeX}{{\sc Bib}\TeX} 
\usepackage{graphicx}
\usepackage{amsfonts}

\setlength{\textheight}{21cm}

\title{{\bf Zadanie nr 3 - Splot, filtracja i korelacja sygnałów}\linebreak
Cyfrowe Przetwarzanie Sygnałów}
\author{Aneta Wiśniewska, 204029 \and Hanna Paluszkiewicz, 203962}
\date{14.05.2018}

\begin{document}
\clearpage\maketitle
\thispagestyle{empty}
\newpage
\setcounter{page}{1}
\section{Cel zadania}

Celem ćwiczenia jest zapoznanie się w praktyce z procesami splotu, filtracji i korelacji sygnałów.

\section{Wstęp teoretyczny}

\subsection{Teoria}


\addtokomafont{labelinglabel}{\sffamily}

\begin{labeling}{alligator}

%%%%%%%%%%%%%%%%%%%%%%%%%%%SPLOT%%%%%%%%%%%%%%%%%%%%%%%%%%%%%%%%%%%%%%%%%%%%%%%%

\item [Splot] to jedno z najważniejszych działań podczas filtracji sygnałów dyskretnych. Jest operacją przetwarzania dwóch sygnałów, w wyniku której otrzymujemy pojedyńczy sygnał dyskretny.  W ogólnym przypadku splot jest zdefiniowany wzorem:
\\
\begin{figure}[h!]
 \centering
 \includegraphics[width=5.3cm]{splotWO.PNG}
 \vspace{-0.3cm}
 \label{Widok_aplikacjis}
\end{figure}
W praktyce stosuje się sygnały o skończonych ilosciach próbek rozmieszczonych równomiernie w dowolnych miejscach osi czasu. Zwykle przyjmuje się konwencję indeksacyjną, gdzie oba sygnały zaczynają się  na osi czasu od próbki zero. Poza granicami przedziału oba sygnały są zerowe. 
\\Wzór dla tej konwencji przyjmuje postać: 
\begin{figure}[h!]
 \centering
 \includegraphics[width=5.3cm]{splotWS.PNG}
 \vspace{-0.3cm}
 \label{Splot_indeks}
\end{figure}

%%%%%%%%%%%%%%%%%%%%%%%%FILTRACJA%%%%%%%%%%%%%%%%%%%%%%%%%%%%%%%%%%%%%%%%%%%%%%%%

\item [Filtracja sygnałów]  - należy do podstawowych operacji CPS. W ramach filtracji widmo sygnału ulega modyfikacji. Zostały odfiltrowane składowe częci sygnału o częstotliwosciach należących do pasma zaporowego. Reszta widma leżąca w pamie przepustowym, nie uległa zmianie lub podlega niewielkiemu tłumieniu. 
\\Filtry ze względu na umiejscowienie pasma przepustowego i zaporowego dzielimy na:
\subitem Filtry dolnoprzepustowe - ich pasma przepustowe są okrelone przedziałem częstotliwosci od 0 do f0 (f0 - częstotliwosć odcięcia filtru) 
\subitem Filtry górnoprzepustowe - ich pasma przepustowe są okrelone przedziałem częstotliwosci od f0 do fp/2 (fp - częstotliwosć próbkowania sygnału) 
\subitem Filtry srodkowoprzepustowe - ich pasma przepustowe są okrelone przedziałem częstotliwosci od fd do fp/4 

\subsubitem f0 = fp/K gdzie f0 to częstotliwosć odcięcia
\subsubitem fp to częstotliwosć 

W zadaniu są stosowane filtry SOI - o skończonej odpowiedzi impulsowej. Zaletą tych filtrów jest łatwosć implementacji (w oparciu o splot) i projektowania postaci filtru. 
\\Przy obliczaniu próbki sygnału wyjsciowego (y(n)) jest brane pod uwagę M przeszłych próbek sygnału wejsciowego (x(n)). Wartosci y(n) obliczamy jako sumy ważone x(n) z uwzględnieniem współczynników filtru h(n). 
\\Opisuje to wzór:

\begin{figure}[h!]
 \centering
 \includegraphics[width=8.3cm]{filtrS.PNG}
 \vspace{-0.3cm}
 \label{filtrS}
\end{figure}

Gdzie:
\subitem M - rząd filtru 
\subitem h(k) - odpowiedź impulsowa

Jeżeli ciąg próbek sygnału wejsciowego będzie ciągiem wartosci zerowych,  filtr SOI będzie generował na wyjsciu skończony ciąg niezerowych wartosci.

Powyższy wzór jest stosowany dla filtru dolnoprzepustowego. Na jego podstawie można obliczyć także działanie innych filtrów. W tym celu stosuje się wzory:
\subsubitem dla filtru srodkowoprzepustowego:
\begin{figure}[h!]
 \centering
 \includegraphics[width=3.3cm]{sr.PNG}
 \vspace{-0.3cm}
 \label{sp}
\end{figure}



\subsubitem dla filtru górnoprzepustowego:
\begin{figure}[h!]
 \centering
 \includegraphics[width=3cm]{gor.PNG}
 \vspace{-0.3cm}
 \label{gp}
\end{figure}


Do projektowania filtrów SOI została zastosowana metoda okna.


W praktyce często stosuje się okna:
\subitem Hamminga
\newpage
\begin{figure}[h!]
 \centering
 \includegraphics[width=6.3cm]{Hm.PNG}
 \vspace{-0.3cm}
 \label{gw}
\end{figure}

\subitem Hanninga

\begin{figure}[h!]
 \centering
 \includegraphics[width=6.3cm]{Hn.PNG}
 \vspace{-0.3cm}
 \label{gw}
\end{figure}

\subitem Blackmana

\begin{figure}[h!]
 \centering
 \includegraphics[width=7.3cm]{B.PNG}
 \vspace{-0.3cm}
 \label{gw}
\end{figure}

%%%%%%%%%%%%%%%%%%%%%%%%KORELACJA%%%%%%%%%%%%%%%%%%%%%%%%%%%%%%%%%%%%%%%%%%%%%%%%

\item [Korelacja] - jest ważną częscią przetwarzania sygnałów. Jest stosowana, gdy trzeba porównać sygnał z innym, zwłaszcza z przesuniętą na osi czasu swoją kopią.
Polega na przetwarzaniu dwóch sygnałów dyskretnych, w czego wyniku otrzymujemy pojedynczy sygnał dyskretny.
\\Korelacja w ogólnym przypadku jest opisywana wzorem:
\begin{figure}[h!]
 \centering
 \includegraphics[width=4.3cm]{kor.PNG}
 \vspace{-0.3cm}
 \label{kr}
\end{figure}
\\Podobnie jak w operacji splotu w praktyce stosuje się sygnały o skończonych ilosciach próbek rozmieszczonych równomiernie w dowolnych miejscach osi czasu i zakres zmiennosci próbek dla każdego n zakresy sumowań zmieniają się zgodnie zumiejscowieniem na osi czasu i liczbami próbek każdego z dyskretnych sygnałów wejsciowych h oraz x. Zwykle przyjmuje się konwencję indeksacyjną, gdzie oba sygnały zaczynają się  na osi czasu od próbki zero. Poza granicami przedziału oba sygnały są zerowe. 
\\Wzór dla tej konwencji przyjmuje postać: 
\begin{figure}[h!]
 \centering
 \includegraphics[width=4.3cm]{korW.PNG}
 \vspace{-0.3cm}
 \label{kw}
\end{figure}


\end{labeling}

%%%%%%%%%%%%%%%%%%%%%%%%INSTRUKCA OBSŁUGI%%%%%%%%%%%%%%%%%%%%%%%%%%%%%%%%%%%%%%%%%%%%

\subsection{Instrukcja obsługi aplikacji}
Aplikacja do generacji szumów zawiera interfejs graficzny, który służy do obsługi przez użytkownika. Wygląd został przedstawiony na poniższym rysunku.
\begin{figure}[h!]
 \centering
 \includegraphics[width=9.3cm]{ui1.PNG}
 \vspace{-0.3cm}
 \caption{Widok główny aplikacji}
 \label{Widok_aplikacjis}
\end{figure}

Na górze okienka znajduje się wysuwana lista możliwych do generacji sygnałów. Obok znajduje się chceckbox, po zaznaczeniu którego sygnał zostanie zapisany do pliku.
Niżej jest przycisk do generacji sygnałów oraz lista parametrów wykresu. Tutaj wpisuje się dane wpływające na sygnał.
Pola umożliwiają ustawienie charakterystycznych parametrów sygnału. Na ich podstawie program wylicza wartości amplitudy sygnału w określonym czasie oraz wyświetla graficzną reprezentację sygnału w postaci wykresu funkcji amplitudy od czasu i histogramu.\\

Na dole okienka znajdują się przyciski do operacji na dwóch sygnałach (dodawanie, odejmowanie itp, a także splot, korelacja). Po kliknięciu w x i y wybieramy odpowiednio pierwszy i dugi składnik działania. Po wcisnięciu przycisku "wynik" program liczy wynik działania i wywietla jego graficzną reprezentację. 
Poniżej znajdują się wysuwane listy z opcjami kwantyzacji, ekstrapolacji. Niżej znajdują się okienka służące do wpisywania parametróW filtracji. Pod nimi znajdują się do wyboru warianty okna i filtru.
Na samym dole okna jest okienko do wpisywania parametrów radaru. 

\subsubsection{Generowanie sygnału}
Aby wygenerować sygnał użytkownik musi kliknąć w generuj sygnał lub w przypadku innych operacji wynik.
\\Po wygenerowaniu sygnału pojawiają się dodatkowe okienko aplikacji.
\begin{figure}[h!]
 \centering
 \includegraphics[width=15.3cm]{prost.PNG}
 \vspace{-0.3cm}
 \caption{Okna po generacji sygnału}
 \label{Widok_aplikacjis}
\end{figure}
\\W kolejnych zakładkach okienka są inne wykresy.

\subsubsection{Odczyt sygnału z pliku}
Oprócz generacji i zapisu do pliku, program umożliwia odczyt z pliku sygnału będącego wynikiem dyskretyzacji (bez kwantyzacji) wygenerowanego
sygnału ciągłego oraz sygnału będącego wynikiem operacji na dwóch sygnałach dyskretnych.
\\Tak jak w przypadku generacji, sygnał jest  reprezentowany graficznie w postaci histogramu i wykresu funkcji.

\subsection{Opis implementacji}
Aplikacja została napisana w wysokopoziomowym języku programowania - C\#. Do rysowania wykresów została wykorzystana zewnątrzna biblioteka OxyPlot. Program został napisany przy pomocy metodyki obiektowej i stosuje metody numeryczne.

\section{Eksperymenty i wyniki}

Poniżej znajdują się wszystkie przeprowadzone eksperymenty - możliwe do uzyskania w aplikacji sygnaly i wyniki. 

%%%%%%%%%%%%%%%%%%%%%%%%%%%%%%%%%%%%%%%%%%%%%%%%%%%%%%%%%%%%%%%%%%%%%%%%%%%%%%%%%%%%%%%%%%%%%%%%%%%%%%%%%%%%%%%%%
% PODROZDZIA PT. EKSPERYMENT NR 1 
%%%%%%%%%%%%%%%%%%%%%%%%%%%%%%%%%%%%%%%%%%%%%%%%%%%%%%%%%%%%%%%%%%%%%%%%%%%%%%%%%%%%%%%%%%%%%%%%%%%%%%%%%%%%%%%%%

\subsection{Eksperyment nr 1}

Eksperyment nr 1 -  Splot\\


\subsubsection{Założenia}
Operacja splotu jest przeprowadzana dla dwóch dowolnych sygnałów dyskretnych o wczeniej podanych (niekonieczcnie jednakowych) ilosciach próbek. W tym celu jest wykorzystany wzór \ref{Splot_indeks}.

\subsubsection{Przebieg}
Do generacji synału zostały podane parametry:
\addtokomafont{labelinglabel}{\sffamily}

\begin{labeling}{szj}
\item [Sygnał 1:] 
\subitem [Amplituda (A):] 1
\subitem [Czas trwania (t1):] 4 s
\subitem [Częstotliwość próbkowania (d): ] 10 Hz
\subitem [Okres podstawowy :] 2 s

\begin{figure}[h!]
 \centering
 \includegraphics[width=12.3cm]{sin.PNG}
 \vspace{-0.3cm}
 \caption{Wykres sygnału sinusoidalnego}
 \label{sin}
\end{figure}

\item [Sygnał 2:] 
\subitem [Amplituda (A):] 1
\subitem [Czas trwania (t1):] 4 s
\subitem [Częstotliwość próbkowania (d): ] 10 Hz

\end{labeling}

\begin{figure}[h!]
 \centering
 \includegraphics[width=12.3cm]{szum.PNG}
 \vspace{-0.3cm}
 \caption{Wykres szumu gausowskiego}
 \label{szum}
\end{figure}
 \newpage
\subsubsection{Rezultat}

Rezultat przedstawia zamieszczony poniżej zrzut ekranu z programu. Wartości liczbowe oraz wykres funkcji amplitudy od czasu przedstawia \ref{Wykres dla wynikw eksperymentu pierwszego}.
\begin{figure}[h!]
 \centering
 \includegraphics[width=12.3cm]{splot.PNG}
 \vspace{-0.3cm}
 \caption{Wykres splotu sygnałów 1 i 2}
 \label{Wykres dla wynikw eksperymentu pierwszego}
\end{figure}


%%%%%%%%%%%%%%%%%%%%%%%%%%%%%%%%%%%%%%%%%%%%%%%%%%%%%%%%%%%%%%%%%%%%%%%%%%%%%%%%%%%%%%%%%%%%%%%%%%%%%%%%%%%%%%%%%
% PODROZDZIA PT. EKSPERYMENT NR2 
%%%%%%%%%%%%%%%%%%%%%%%%%%%%%%%%%%%%%%%%%%%%%%%%%%%%%%%%%%%%%%%%%%%%%%%%%%%%%%%%%%%%%%%%%%%%%%%%%%%%%%%%%%%%%%%%%

\subsection{Eksperyment nr 2}
Eksperyment nr 2  - Filtracja z filtrem dolnoprzepustowym

\subsubsection{Założenia}
Filtr dolnoprzepustowy opisuje wzór, na podstawie odwrotnego przekształcenia Fouriera:

\begin{figure}[h!]
 \centering
 \includegraphics[width=6.3cm]{four.PNG}
 \vspace{-0.3cm}
 \label{gw}
\end{figure}

gdzie:
\subitem n - liczba całkowita,
\subitem częstotliwosć odcięcia filtru - f0 =  fp/K

Zakładamy, że filtr jest idealny -  w pasmie przepustowym nie zmienia się widmo sygnału wejsciowego -  transmitancja jest równa  1. W pasmie zaporowym skłądowe czętotliwosciowe zostaną kompletnie wytłumione (transmitancja równa 0).

Ze względu na nieskońconą liczbę współczynników h(n) nie stosuje się tego wzoru w praktyce.

Wzór na odpowiedź impulsową filtru o M współczynników z przesunięciem (w celu uzyskania nieujemnych indeksów):

\begin{figure}[h!]
 \centering
 \includegraphics[width=6.3cm]{f.PNG}
 \vspace{-0.3cm}
 \label{gw}
\end{figure}

gdzie:
\subitem n = 0,1, ... , M-1
\subitem częstotliwosć odcięcia filtru - f0 =  fp/K

\subsubsection{Przebieg}
Do generacji synału prostokątnego zostały podane parametry:
\addtokomafont{labelinglabel}{\sffamily}

\begin{labeling}{szj}
\item [Amplituda (A):] 1
\item [Czas trwania (t1):] 100 s
\item [Częstotliwość próbkowania (d): ] 1 Hz
\item [Okres podstawowy :] 100 s
\item [Współczynnik wypełnienia:] 0,5
\end{labeling}


1:
\begin{labeling}{szj}
\item [K:] 10
\item [M:] 20 
\end{labeling}

2: (drugi przykład filtru dolnoprzepustowego i okna prostokątnego)

\begin{labeling}{szj}
\item [K:] 10
\item [M:] 100 
\end{labeling}

3: (trzeci przykład filtru dolnoprzepustowego i okna prostokątnego)

\begin{labeling}{szj}
\item [K:] 100
\item [M:] 20 
\end{labeling}
\subsubsection{Rezultat}

Rezultaty przedstawiają zamieszczone poniżej zrzuty ekranu z programu. 

Okno prostokątne
\\1:
\begin{figure}[h!]
 \centering
 \includegraphics[width=12.3cm]{prostFDOP.PNG}
 \vspace{-0.3cm}
 \caption{Filtracja dolnoprzepustowa z oknem prostokątnym}
 \label{Wykres dla wyników eksperymentu drugiego}
\end{figure}
\newpage
Rys. \ref{Wykres dla wynikw eksperymentu pierwszego h} przedstawia histogram sygnału z opisanymi powyżej parametrami. 
\begin{figure}[h!]
 \centering
 \includegraphics[width=12.3cm]{prostSFDP.PNG}
 \vspace{-0.3cm}
 \caption{Sygnał filtracji dolnoprzepustowej z oknem prostokątnym}
 \label{Histogram dla wyników eksperymentu drugiego}
\end{figure}
2:
\begin{figure}[h!]
 \centering
 \includegraphics[width=12.3cm]{filtrMk.PNG}
 \vspace{-0.3cm}
 \caption{Filtracja dolnoprzepustowa z oknem prostokątnym}
 \label{Wykres dla wyników eksperymentu drugiego}
\end{figure}
\newpage

3:
\begin{figure}[h!]
 \centering
 \includegraphics[width=12.3cm]{dK.PNG}
 \vspace{-0.3cm}
 \caption{Filtracja dolnoprzepustowa z oknem prostokątnym}
 \label{pdg}
\end{figure}
\newpage

\begin{figure}[h!]
 \centering
 \includegraphics[width=12.3cm]{filtrMkS.PNG}
 \vspace{-0.3cm}
 \caption{Sygnał filtracja dolnoprzepustowej z oknem prostokątnym}
 \label{filtrMks}
\end{figure}

Okno Hamminga
\begin{figure}[h!]
 \centering
 \includegraphics[width=12.3cm]{prostFDOHm.PNG}
 \vspace{-0.3cm}
 \caption{Filtracja dolnoprzepustowa z oknem Hamminga}
 \label{fdohm}
\end{figure}


\begin{figure}[h!]
 \centering
 \includegraphics[width=12.3cm]{prostSFDHm.PNG}
 \vspace{-0.3cm}
 \caption{Sygnał filtracja dolnoprzepustowej z oknem Hamminga}
 \label{hm}
\end{figure}

\newpage
Okno Hanninga
\begin{figure}[h!]
 \centering
 \includegraphics[width=12.3cm]{prostFDOHn.PNG}
 \vspace{-0.3cm}
 \caption{Filtracja dolnoprzepustowa z oknem Hanninga}
 \label{Wykres dla wyników eksperymentu drugiego}
\end{figure}


\begin{figure}[h!]
 \centering
 \includegraphics[width=12.3cm]{prostSFDHn.PNG}
 \vspace{-0.3cm}
 \caption{Sygnał filtracji dolnoprzepustowej z oknem Hanninga}
 \label{hn}
\end{figure}

\newpage
Okno Blackmana
\begin{figure}[h!]
 \centering
 \includegraphics[width=12.3cm]{prostFDOB.PNG}
 \vspace{-0.3cm}
 \caption{Filtracja dolnoprzepustowa z oknem Blackmana}
 \label{bm}
\end{figure}
\newpage
\begin{figure}[h!]
 \centering
 \includegraphics[width=12.3cm]{prostSFDB.PNG}
 \vspace{-0.3cm}
 \caption{Sygnał filtracji dolnoprzepustowej z oknem Blackmana}
 \label{sb}
\end{figure}

%%%%%%%%%%%%%%%%%%%%%%%%%%%%%%%%%%%%%%%%%%%%%%%%%%%%%%%%%%%%%%%%%%%%%%%%%%%%%%%%%%%%%%%%%%%%%%%%%%%%%%%%%%%%%%%%%
% PODROZDZIA PT. EKSPERYMENT NR 3
%%%%%%%%%%%%%%%%%%%%%%%%%%%%%%%%%%%%%%%%%%%%%%%%%%%%%%%%%%%%%%%%%%%%%%%%%%%%%%%%%%%%%%%%%%%%%%%%%%%%%%%%%%%%%%%%%

\subsection{Eksperyment nr 3}

Eksperyment nr 3  - Filtracja z filtrem srodkowoprzepustowym
\subsubsection{Założenia}
Z wykorzystaniem twierdzenia o modulacji przekształcamy odpowiedź impulsową filtru dolnoprzepustowego do odpowiedzi filtru srodkowoprzepustowego:
współczynniki h(n) są mnożone przez sygnał sinusoidalny  o częstotliwosci f = fp/4

Wtedy fd = fp/4-f0 i fg = fp/4 + f0.

\subsubsection{Przebieg}
Do generacji synału prostokątnego zostały podane parametry:
\addtokomafont{labelinglabel}{\sffamily}

\begin{labeling}{szj}
\item [Amplituda (A):] 1
\item [Czas trwania (t1):] 100 s
\item [Częstotliwość próbkowania (d): ] 1 Hz
\item [Okres podstawowy :] 100 s
\item [Współczynnik wypełnienia:] 0,5
\end{labeling}

Wykres sygnału przedstawia poniższy obrazek:
\begin{figure}[h!]
 \centering
 \includegraphics[width=12.3cm]{prost.PNG}
 \vspace{-0.3cm}
 \label{gw}
\end{figure}

Parametry filtracji:
\addtokomafont{labelinglabel}{\sffamily}

\begin{labeling}{szj}
\item K:] 10
\item [M:] 20 s
\end{labeling}

\subsubsection{Rezultat}

Rezultaty przedstawiają zamieszczone poniżej zrzuty ekranu z programu. 

Okno prostokątne
\begin{figure}[h!]
 \centering
 \includegraphics[width=12.3cm]{prostFSOP.PNG}
 \vspace{-0.3cm}
 \caption{Filtracja srodkowoprzepustowa z oknem prostokątnym}
 \label{Wykres dla wyników eksperymentu drugiego}
\end{figure}

\begin{figure}[h!]
 \centering
 \includegraphics[width=12.3cm]{prostSFSP.PNG}
 \vspace{-0.3cm}
 \caption{Sygnał filtracji srodkowoprzepustowej z oknem prostokątnym}
 \label{sfsp}
\end{figure}

\newpage
Okno Hamminga
\begin{figure}[h!]
 \centering
 \includegraphics[width=12.3cm]{prostFSOHm.PNG}
 \vspace{-0.3cm}
 \caption{Filtracja srodkowoprzepustowa z oknem Hamminga}
 \label{fsohm}
\end{figure}

\newpage
\begin{figure}[h!]
 \centering
 \includegraphics[width=12.3cm]{prostSFSHm.PNG}
 \vspace{-0.3cm}
 \caption{Sygnał filtracji srodkowoprzepustowej z oknem Hamminga}
 \label{sfshm}
\end{figure}

Okno Hanninga
\begin{figure}[h!]
 \centering
 \includegraphics[width=12.3cm]{prostFSOHn.PNG}
 \vspace{-0.3cm}
 \caption{Filtracja srodkowoprzepustowa z oknem prostokątnym}
 \label{Wykres dla wyników eksperymentu drugiego}
\end{figure}

\newpage
\begin{figure}[h!]
 \centering
 \includegraphics[width=12.3cm]{prostSFSHn.PNG}
 \vspace{-0.3cm}
 \caption{Sygnał filtracji srodkowoprzepustowej z oknem Hamminga}
 \label{sfshn}
\end{figure}

Okno Blackmana
\begin{figure}[h!]
 \centering
 \includegraphics[width=12.3cm]{prostFSOB.PNG}
 \vspace{-0.3cm}
 \caption{Filtracja srodkowoprzepustowa z oknem prostokątnym}
 \label{fsob}
\end{figure}

\newpage

\begin{figure}[h!]
 \centering
 \includegraphics[width=12.3cm]{prostSFSB.PNG}
 \vspace{-0.3cm}
 \caption{Sygnał filtracji srodkowoprzepustowej z oknem Blackmana}
 \label{ob}
\end{figure}


%%%%%%%%%%%%%%%%%%%%%%%%%%%%%%%%%%%%%%%%%%%%%%%%%%%%%%%%%%%%%%%%%%%%%%%%%%%%%%%%%%%%%%%%%%%%%%%%%%%%%%%%%%%%%%%%%
% PODROZDZIA PT. EKSPERYMENT NR 4 
%%%%%%%%%%%%%%%%%%%%%%%%%%%%%%%%%%%%%%%%%%%%%%%%%%%%%%%%%%%%%%%%%%%%%%%%%%%%%%%%%%%%%%%%%%%%%%%%%%%%%%%%%%%%%%%%%

\subsection{Eksperyment nr 4}

Eksperyment nr 4  - Filtracja z filtrem górnoprzepustowym
\subsubsection{Założenia}
Z wykorzystaniem twierdzenia o modulacji przekształcamy odpowiedź impulsową filtru dolnoprzepustowego do odpowiedzi filtru górnoprzepustowego:
współczynniki h(n) są mnożone przez sygnał sinusoidalny  o częstotliwosci f = fp/2. Wtedy f0 = fp/2-f0 (f0 -  nowa częstotliwosć odcięcia).

\subsubsection{Przebieg}
Do generacji synału prostokątnego zostały podane parametry:
\addtokomafont{labelinglabel}{\sffamily}

\begin{labeling}{szj}
\item [Amplituda (A):] 1
\item [Czas trwania (t1):] 100 s
\item [Częstotliwość próbkowania (d): ] 1 Hz
\item [Okres podstawowy :] 100 s
\item [Współczynnik wypełnienia:] 0,5
\end{labeling}

Wykres sygnału przedstawia poniższy obrazek:
\begin{figure}[h!]
 \centering
 \includegraphics[width=12.3cm]{prost.PNG}
 \vspace{-0.3cm}
 \label{gw}
\end{figure}

Parametry filtracji:
\addtokomafont{labelinglabel}{\sffamily}

\begin{labeling}{szj}
\item K:] 10
\item [M:] 20 s
\end{labeling}

\subsubsection{Rezultat}

Rezultaty przedstawiają zamieszczone poniżej zrzuty ekranu z programu. 
Okno prostokątne
\begin{figure}[h!]
 \centering
 \includegraphics[width=12.3cm]{prostFGOP.PNG}
 \vspace{-0.3cm}
 \caption{Filtracja górnoprzepustowa z oknem prostokątnym}
 \label{gfh}
\end{figure}
 
\begin{figure}[h!]
 \centering
 \includegraphics[width=12.3cm]{prostSFGP.PNG}
 \vspace{-0.3cm}
 \caption{Sygnał filtracji górnoprzepustowej z oknem prostokątnym}
 \label{hg}
\end{figure}

\newpage
Okno Hamminga
\begin{figure}[h!]
 \centering
 \includegraphics[width=12.3cm]{prostFGOHm.PNG}
 \vspace{-0.3cm}
 \caption{Filtracja górnoprzepustowa z oknem Hamminga}
 \label{dtf}
\end{figure}
\newpage

\begin{figure}[h!]
 \centering
 \includegraphics[width=12.3cm]{prostSFGHm.PNG}
 \vspace{-0.3cm}
 \caption{Sygnał filtracji górnoprzepustowej z oknem Hamminga}
 \label{esrd}
\end{figure}

Okno Hanninga
\begin{figure}[h!]
 \centering
 \includegraphics[width=12.3cm]{prostFGOHn.PNG}
 \vspace{-0.3cm}
 \caption{Filtracja górnoprzepustowa z oknem Hanninga}
 \label{seaw}
\end{figure}
\newpage

\begin{figure}[h!]
 \centering
 \includegraphics[width=12.3cm]{prostSFGHn.PNG}
 \vspace{-0.3cm}
 \caption{Sygnał filtracji górnoprzepustowej z oknem Hanninga}
 \label{qwe}
\end{figure}

Okno Blackmana
\begin{figure}[h!]
 \centering
 \includegraphics[width=12.3cm]{prostFGOB.PNG}
 \vspace{-0.3cm}
 \caption{Filtracja górnoprzepustowa z oknem Blackmana}
 \label{werty}
\end{figure}
\newpage

\begin{figure}[h!]
 \centering
 \includegraphics[width=12.3cm]{prostSFGB.PNG}
 \vspace{-0.3cm}
 \caption{Sygnał filtracji górnoprzepustowa z oknem Blackmana}
 \label{joi}
\end{figure}


%%%%%%%%%%%%%%%%%%%%%%%%%%%%%%%%%%%%%%%%%%%%%%%%%%%%%%%%%%%%%%%%%%%%%%%%%%%%%%%%%%%%%%%%%%%%%%%%%%%%%%%%%%%%%%%%%
% PODROZDZIA PT. EKSPERYMENT NR 5
%%%%%%%%%%%%%%%%%%%%%%%%%%%%%%%%%%%%%%%%%%%%%%%%%%%%%%%%%%%%%%%%%%%%%%%%%%%%%%%%%%%%%%%%%%%%%%%%%%%%%%%%%%%%%%%%%

\subsection{Eksperyment nr 5}

Eksperyment nr 5 - Korelacja\\

\subsubsection{Założenia}
Eksperyment jest przeprowadzany na dwa sposoby: implementację bezporednią i implementację z użyciem splotu.
\\W pierwszysm przypadku stosujemy do operacji korelacji wzór bezposredni   \ref{kw}. W drugim do obliczenia korelacji jest wykorzystywany wzór na splot \ref{Splot_indeks}.

\subsubsection{Przebieg}
Do generacji synału zostały podane parametry:
\addtokomafont{labelinglabel}{\sffamily}

\begin{labeling}{szj}
\item [Sygnał 1:]
\subitem [Amplituda (A):] 1
\subitem [Czas trwania (t1):] 5 s
\subitem [Częstotliwość próbkowania (d): ] 10 Hz
\subitem [Okres podstawowy :] 2 s

\begin{figure}[h!]
 \centering
 \includegraphics[width=12.3cm]{sin.PNG}
 \vspace{-0.3cm}
 \caption{Wykres sygnału sinusoidalnego}
 \label{sin}
\end{figure}

\item [Sygnał 2:]
\subitem [Amplituda (A):] 1
\subitem [Czas trwania (t1):] 5 s
\subitem [Częstotliwość próbkowania (d): ] 10 Hz

\end{labeling}

\begin{figure}[h!]
 \centering
 \includegraphics[width=12.3cm]{szum.PNG}
 \vspace{-0.3cm}
 \caption{Wykres szumu gausowskiego}
 \label{szum}
\end{figure}
 \newpage
\subsubsection{Rezultat}
\newpage
Korelacja bezposrednia:
\\Rezultat przedstawia zamieszczony poniżej zrzut ekranu z programu.  Został zastosowany wzór\ref{kw}.
\begin{figure}[h!]
 \centering
 \includegraphics[width=12.3cm]{korB.PNG}
 \vspace{-0.3cm}
 \caption{Wykres korelacji bezporedniej}
 \label{wk}
\end{figure}

Korelacja z wykorzystaniem splotu:
\\Rezultat przedstawia zamieszczony poniżej zrzut ekranu z programu.  Został zastosowany wzór\ref{Splot_indeks}.
\begin{figure}[h!]
 \centering
 \includegraphics[width=12.3cm]{korS.PNG}
 \vspace{-0.3cm}
 \caption{Wykres korelacji z wykorzystaniem splotu}
 \label{redtrfytguyhiuj}
\end{figure}

%%%%%%%%%%%%%%%%%%%%%%%%%%%%%%%%%%%%%%%%%%%%%%%%%%%%%%%%%%%%%%%%%%%%%%%%%%%%%%%%%%%%%%%%%%%%%%%%%%%%%%%%%%%%%%%%%
% PODROZDZIA PT. EKSPERYMENT NR 6
%%%%%%%%%%%%%%%%%%%%%%%%%%%%%%%%%%%%%%%%%%%%%%%%%%%%%%%%%%%%%%%%%%%%%%%%%%%%%%%%%%%%%%%%%%%%%%%%%%%%%%%%%%%%%%%%%

\subsection{Eksperyment nr 6}

Eksperyment nr 6 - Radar\\

\subsubsection{Założenia}
Analiza korelacyjna (porównywanie sygnałów) jest stosowana do pomiaru odległosci sygnałów przesuniętych w czasie od celu. Stosuje się do tego radar. Wysyła on sygnał sondujący, który w okrelonych przypadkach (gdy sygnał jest odpowiednio zmodulowany) może być sygnałem okresowym.
Sygnał po odbiciu się od celu wraca do nadajnika.
\\Pomiar odległosci dokonuje się na bazie pomiaru opóźnienia przy użyciu analizy korelacyjnej sygnału wysłanego i powracającego.
Gdy częstotliwosć próbkowania obu sygnałów jest jednakowa i odpowiednio duża, można w wyznaczonych odstępach dokonywać analizy korelacyjnej, która jest spróbkowaana i zbuforowana, dwóch sygnałów -  sondującego i zwrotnego. Osiąga się to dzięki obliczeniu wzajemnej korelacji pary odpowaiadających sobie sygnałów, w celu uaktualnienia odczytu odległosci od celu. 

Korelacja przyjmuje największą wartosć, gdy nakładane na siebie sygnały pokrywają się w jak największym stopniu. Wzór \ref{kw} umożliwia nałożenie obydwu
sygnałów dla każdego odstępu czasowego próbkowania, z odpowiednim przesunięciem sygnałów sondującego i zwrotnego względem siebie. Także pozwala nam na obliczenie pojedyńczej wartosci korelacji dla każdego z odstępów. Możemy znaleźć maksimum funkcji korelacji. 
 

\subsubsection{Przebieg}
Została zaimplementowana symulacja działania korelacyjnego czujnika odległosci.

Do generacji synału prostokątnego zostały podane parametry:
\addtokomafont{labelinglabel}{\sffamily}

\begin{labeling}{szj}
\item [Amplituda (A):] 1
\item [Czas trwania (t1):] 100 s
\item [Częstotliwość próbkowania (d): ] 1 Hz
\item [Okres podstawowy :] 25 s
\end{labeling}

Wykres sygnału przedstawia poniższy obrazek:
\newpage
\begin{figure}[h!]
 \centering
 \includegraphics[width=12.3cm]{sin1.PNG}
 \vspace{-0.3cm}
 \label{gw}
\end{figure}
\subsubsection{Rezultat}

Opóźnienie = 5:
\\Rezultat przedstawia zamieszczony poniżej zrzut ekranu z programu. Wartości liczbowe oraz wykres funkcji amplitudy od czasu przedstawia \ref{Wykres dla wynikw eksperymentu pierwszego}.
\begin{figure}[h!]
 \centering
 \includegraphics[width=12.3cm]{sin1R5.PNG}
 \vspace{-0.3cm}
 \caption{Radar sygnaűłu sinusoidalnego z opóźnieniem 5}
 \label{R5}
\end{figure}


Opóźnienie = 20:
\\Rezultat przedstawia zamieszczony poniżej zrzut ekranu z programu. Wartości liczbowe oraz wykres funkcji amplitudy od czasu przedstawia \ref{Wykres dla wynikw eksperymentu pierwszego}.
\begin{figure}[h!]
 \centering
 \includegraphics[width=12.3cm]{sin1R20.PNG}
 \vspace{-0.3cm}
 \caption{Radar sygnału sinusoidalnego z opóźnieniem 20}
 \label{r20}
\end{figure}

%%%%%%%%%%%%%%%%%%%%%%%%%%%%%%%%%%%%%%%%%%%%%%%%%%%%%%%%%%%%%%%%%%%%%%%%%%%
% PODROZDZIA PT. WNIOSKI
%%%%%%%%%%%%%%%%%%%%%%%%%%%%%%%%%%%%%%%%%%%%%%%%%%%%%%%%%%%%%%%%%%%%%%%%%%%

\section{Wnioski}
\subsection{Splot}
Splot dwóch funkcji w dziedzinie czasu jest równoważny pomnożeniu ich widm w dziedzinie częstotliwości. Można też wyrazić go przez korelacje - trzeba sygnał x odwrocić. Długosć splotu równa się sumie dlugosci sygnałów splotowanych -1.
W wyniku operacji splotu sygnały zostały nałożone na siebie i w wyniku zastosowania odpowiednich wzorów sygnał wyjsciowy ma cechy obu sygnalów.
\subsection{Filtracja}
W filtrze dolnoprzepustowym okna wygladzaj filtr a zatem filtrowany sygnał i wyjsciowy są zbliżone (sa niewielkie roznice).
Parametr K wplywa na częstotliwosć odcięcia. Przy większym M, czyli rzędzie, filtr jest dłuższy oraz pojawia się na nim wiecej zmian wartosci (zafalowań).
\\ Z filtrami srodkowoprzepustowymi sytuacja z oknami wygląda podobnie jak w filtrze dolnoprzepustowym.


Różne typow filtrów odcinają czestotliwosci w innych pasmach. W przeciwienstwie do filtru dolnoprzepustowego, filtr srodkowoprzepustowy często zmienia swoje wartosci, a  w filtrze górnoprzepustowym co druga próbka jest ujemna. W filtrze dolnoprzepustowym wykres filtru jest gładki.
Nie możemy wiele powiedzieć o filtrach, ze względu na brak informacji  o widmie częstotliwosciowym.
Stosując okna zmniejszamy zafalowania charakterystyki. 
Z analizy pokazanych wykresów można stwierdzić, że okna Hanninga, Hamminga i Blackmana w dużym stopniu redukują nieliniowosć transmitancji w pamie przepustowym. Dodatkowa istotnie polepszają tłumienie składowych w pamie zaporowym w odniesieniu do okna prostokątnego.


\subsection{Korelacja}
Korelację można wykonać splotem i tez trzeba sygnał x odwrocić.
Z eksperymentów wynika, że korelacja bezporednia i z wykorzystaniem splotu dają bardzo zbliżone efekty.


\subsection{Radar}
Z eksperymentów wynika, że jesli zalożymy taką samą częstotliwosć próbkowania, to wieksze opóźnienie równa się większej odleglosci. Poniewaz są t0 sygnały okresowe, to przyjmując okres = np 25 opóznienie równe 5 oraz 30 bedą dawaly takie same wyniki.
Ponieważ przyjęłysmy jako predkosć predkosć swiatla to odleglosci wychodzą bardzo duże.
Trzeba pamietać, że czestotliwosć próbkowania nie zawsze będzie wynosić 1Hz czyli 1 próbke na sekunde - przy obliczeniu odstepu czasowego zawsze trzeba otrzymywać wynik w sekundach i podzielić 1/czestotliwosć.

%%%%%%%%%%%%%%%%%%%%%%%%%%%%%%%%%%%%%%%%%%%%%%%%%%%%%%%%%%%%%%%%%%%%%%%%%%%
% PODROZDZIA PT. ZALACZNIKI
%%%%%%%%%%%%%%%%%%%%%%%%%%%%%%%%%%%%%%%%%%%%%%%%%%%%%%%%%%%%%%%%%%%%%%%%%%%
\begin{thebibliography}{0}
 \bibitem{l2short} FTIMS Politechnika Łódzka.
    \textsl{Zadanie 3 Splot, filtracja i korelacja}, Wikamp.
\end{thebibliography}

\end{document}